\chapter{Instrukcja użycia biblioteki w celu wykonania zapytania}
Po dodaniu biblioteki \texttt{Thesis.Relinq} do własnego projektu i dodaniu zależności do niej w kodzie klienckim (\texttt{using Thesis.Relinq}), dla każdej tabeli w bazie danych, z której użytkownik chce wyciągać dane, należy napisać klasę modelową, która jest udekorowana atrybutami \texttt{Table} i \texttt{Column} określającymi nazwę tabeli i nazwy jej kolumn w bazie danych. Przykładowa klasa modelowa wygląda następująco:

\begin{lstlisting}
[Table(Name = "people")]
public class Person
{
    [Column(Name = "person_id")]
    public int PersonId { get; set; }

    [Column(Name = "first_name")]
    public string FirstName { get; set; }

    [Column(Name = "last_name")]
    public string LastName { get; set; }
}
\end{lstlisting}

Po opisaniu każdej z tabel, źródło danych w zapytaniu LINQ jest tworzone za pomocą metody \texttt{PsqlQueryFactory.Queryable<T>(DbConnection connection)}, która zwraca obiekty \texttt{PsqlQueryable}. Obiekt \texttt{DbConnection} powinien być kompatybilny z bazą PostgreSQL oraz zostać wcześniej z taką bazą połączony.

Wszystkie obiekty \texttt{PsqlQueryable} można opakować w klasę kontekstową, która pozwola na o wiele czytelniejsze wykonywanie zapytań. Przykładowa klasa kontekstowa oraz jej wykorzystanie w zapytaniu wygląda następująco:

\pagebreak

\begin{lstlisting}
public class NorthwindContext
{
    public PsqlQueryable<Order> Orders { get; }
    public PsqlQueryable<Customer> Customers { get; }
    public PsqlQueryable<Employee> Employees { get; }

    public NorthwindContext(DbConnection connection)
    {
        this.Customers = PsqlQueryFactory
            .Queryable<Customer>(connection);
        this.Employees = PsqlQueryFactory
            .Queryable<Employee>(connection);
        this.Orders = PsqlQueryFactory
            .Queryable<Order>(connection);
    }
}


var connection = new SomePostgresConnection();
var context = new NorthwindContext(connection);

var myQuery =
    from c in Context.Customers
    join o in Context.Orders
    on c.CustomerID equals o.CustomerID
    select new
    {
        Name = c.ContactName,
        Order = o.OrderID
    };
\end{lstlisting}