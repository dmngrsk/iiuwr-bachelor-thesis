% Opcje klasy 'iithesis' opisane sa w komentarzach w pliku klasy. Za ich pomoca
% ustawia sie przede wszystkim jezyk i rodzaj (lic/inz/mgr) pracy, oraz czy na
% drugiej stronie pracy ma byc skladany wzor oswiadczenia o autorskim wykonaniu.
\documentclass[declaration,shortabstract,polish,inz]{iithesis}
\usepackage[utf8]{inputenc}
\usepackage{fancyhdr}
\usepackage[toc]{appendix}
\usepackage{bookmark}
\usepackage{enumerate}
\usepackage{csvsimple}

\usepackage{color}
\definecolor{bluekeywords}{rgb}{0.13,0.13,1}
\definecolor{greencomments}{rgb}{0,0.5,0}
\definecolor{redstrings}{rgb}{0.9,0,0}

\usepackage{listings}
\lstset{language=[Sharp]C,
  showspaces=false,
  showtabs=false,
  breaklines=true,
  showstringspaces=false,
  breakatwhitespace=true,
  escapeinside={(*@}{@*)},
  morekeywords={var, from, in, where, select, join, orderby, descending, get, set},
  commentstyle=\color{greencomments},
  keywordstyle=\color{bluekeywords},
  stringstyle=\color{redstrings},
  basicstyle=\ttfamily
}

\pagestyle{fancy}
\fancyhf{}
\fancyfoot[CO,CE]{\thepage}
\fancyhead[RE]{\leftmark}
\fancyhead[LO]{\rightmark}
\renewcommand{\chaptermark}[1]{\markboth{Damian Górski, Instytut Informatyki UWr}{}}
\renewcommand{\sectionmark}[1]{\markright{Implementacja języka zapytań oparta na drzewach rozbioru}}

\polishtitle{Implementacja języka zapytań\fmlinebreak oparta na drzewach rozbioru}
\englishtitle{Implementation of a query language\fmlinebreak based on partition trees}

\author{Damian Górski}
\advisor{dr Wiktor Zychla}
\date{30 czerwca 2017 r.} % Data zlozenia pracy
\transcriptnum{273212} % Numer indeksu
\advisorgen{dr Wiktora Zychli} % Nazwisko promotora w dopelniaczu

\polishabstract{W czasie projektowania pewnego systemu informatycznego, architekt zazwyczaj musi zmierzyć się z wyborem bazy danych, którą chce użyć w danej aplikacji. Problem pojawia się w momencie, gdy pewna baza danych oferuje to, czego szuka, ale potrzebuje sposobu wybierania z niej informacji w wygodny dla programistów sposób. W technologii .NET pozwala na to Language INtegrated Query (LINQ), który tłumaczy zapytanie w swoim języku na drzewo rozbioru składniowego, po którym można przejść implementując pewien zbiór klas odwiedzających to drzewo (visitorów), w celu stworzenia zapytania w docelowej bazie danych. Tematem niniejszej pracy jest implementacja klas, które pozwolą na stworzenie dostawcy LINQ dla bazy PostgreSQL.}

\englishabstract{While projecting a computer system, we frequently have to cope with the task of choosing the database we want to use in our application. The problem is, some databases offer what we need, but we also need a more comfortable way to obtain information from it. The .NET framework allows us to achieve that with the Language INtegrated Query (LINQ) technology, which translates a query in its language into an abstract syntax partition tree, that we can traverse by implementing a set of visitors in order to create a query to our target database. The topic of this thesis is the implementation of such visitors, that will allow to build a LINQ provider for the PostgreSQL database.}

\begin{document}

\chapter{Preliminaria}
W celu zrozumienia mechanizmu budowy zapytania SQL-owego, trzeba najpierw zrozumieć sposób działania języka LINQ, który jest punktem wejścia, oraz struktury drzewa rozbioru składniowiowego, będącym przedmiotem translacji \linebreak LINQ-to-SQL. Zakładam, że czytelnikowi znane są podstawowe pojęcia związane z programowaniem obiektowym, takie jak metoda, kolekcja, dziedziczenie, typ generyczny. W niniejszym rozdziale poruszone zostaną następujące tematy:

\begin{itemize}
\item Sposób przetrzymywania kolekcji wyliczalnych w .NET-cie.
\item Opis i motywacja powstania języka zapytań LINQ.
\item Struktura drzewa wyrażeń \texttt{IQueryable}, i dlaczego takie drzewa są trudne do odwiedzania w celu zrealizowania zadania LINQ-to-SQL.
\item Biblioteka re-linq uproszczająca powyższe drzewa, obiekty \texttt{QueryModel}.
\end{itemize}

\section{Słowo o \texttt{IEnumerable<T>} i \texttt{IQueryable<T>}}
We frameworku .NET wszystkie kolekcje, które możemy wyliczyć (a takie nas interesują, bo pracujemy z relacyjną bazą danych), implementują interfejs \linebreak \texttt{IEnumerable<T>}, gdzie \texttt{T} jest typem obiektu, który jest przetrzymywany w kolekcji. Ten interfejs definiuje metodę \texttt{GetEnumerator()}, który zwraca obiekt typu \texttt{IEnumerator<T>}, który ma właściwość \texttt{Current} oraz metodę \texttt{MoveNext()}, pozwalając na przejście po uporządkowanym ciągu obiektów typu \texttt{T} oraz określenie obecnej pozycji. Korzystając z tych dwóch informacji, jesteśmy w stanie rozszerzyć \texttt{IEnumerable<T>} o metody takie jak wyznaczenie długości, filtrowanie kolekcji, łączenie dwóch kolekcji ze sobą, mapowanie funkcji na wszystkie obiekty znajdujące się w kolekcji. Dokładna lista metod rozszerzających \texttt{IEnumerable<T>} jest dostępna \href{https://msdn.microsoft.com/pl-pl/library/9eekhta0(v=vs.110).aspx}{w oficjalnej dokumentacji MSDN}. \pagebreak

Rozszerzeniem \texttt{IEnumerable<T>} jest interfejs \texttt{IQueryable<T>}, który de facto implementuje \texttt{IEnumerable<T>}. Zasadniczą różnicą między tymi dwoma interfejsami jest to, że w momencie wywołania ciągu metod rozszerzających \texttt{IEnumerable<T>}, każda z tych metod jest wywoływana jedna po drugiej, co może obciążyć moc obliczeniową procesora. Natomiast kolekcja \texttt{IQueryable<T>} jest świadoma, że nie musi wykonywać tych metod od razu, tylko przetrzymuje je w postaci drzewa wyrażeń (o wyrażeniach w następnej sekcji), które dopiero przy wywołaniu metody wyliczającej elementy z kolekcji zostaje wykonane w całości w efektywny sposób. Takie rozwiązanie jest idealne dla kolekcji, które łączą się z zewnętrzną bazą danych, aby istniała możliwość wybrania danych za pomocą jednego dużego zapytania SQL-owego.

\section{Language INtegrated Query}
Programiści na codzień pracują z danymi w różnych formach - zapisanych w plikach XML i JSON, przetrzymywanych w bazie danych, czy też po prostu z kolekcjami obiektów. Nie jest sztuką zauważyć, że trudnością dla programisty będzie odnalezienie się w projekcie, który korzysta z wielu źródeł danych, ponieważ wybranie danych z każdego z nich wymaga znajomości metod używania tych źródeł. To dało do myślenia architektom z Microsoftu, którzy ,,postanowili uogólnić problem [wyboru danych] i dodać możliwość wykonywania zapytań w sposób kompatybilny ze wszystkimi źródłami danych, nie tylko relacyjnymi i XML-owymi. Rozwiązanie to nazwali \textbf{L}anguage \textbf{IN}tegrated \textbf{Q}uery'' \cite{msdn_linq}, i zostało bardzo ciepło przyjęte przez programistów .NET. Zapytanie LINQ jest automatycznie tłumaczone do docelowego języka zapytań, którego programista C\# lub VB nie musi znać - a więc jest w stanie wybierać dane z niemal każdego źródła z użyciem tej samej składni.

Poniżej zostało przedstawione przykładowe zapytanie LINQ, które wybiera \linebreak imiona i nazwiska osób z kolekcji pracowników, którzy zarabiają więcej niż 3000 złotych, posortowane alfabetycznie po nazwiskach:

\begin{lstlisting}
var linqQuery = 
    from e in db.Employees
    where e.Salary > 3000.0
    orderby e.LastName
    select new
    { 
        FirstName = e.FirstName,
        LastName = e.LastName
    };
\end{lstlisting}

Takie zapytanie można również zapisać za pomocą metod z użyciem wyrażeń lambda (powyższe zapytanie jest tłumaczone przez kompilator do poniższego): \pagebreak

\begin{lstlisting}
var linqQuery2 = db.Employees
    .Where(e => e.Salary > 3000.0)
    .OrderBy(e => e.LastName)
    .Select(e => new
    { 
        FirstName = e.FirstName,
        LastName = e.LastName
    });
\end{lstlisting}

Pisząc zapytanie LINQ, tak naprawdę wykonywane są metody na kolekcjach \texttt{IEnumerable<T>}, z którymi była okazja zapoznać się w trakcie czytania sekcji traktującej o kolekcjach, które implementują ten interfejs. Każde z tych zapytań zwraca kolekcję \texttt{IEnumerable<T>} (w przypadku danych wybieranych z zewnętrznego źródła - \texttt{IQueryable<T>}), gdzie \texttt{T} jest typem anonimowym zawierającym dwie właściwości \texttt{FirstName} i \texttt{LastName}. Tą kolekcję można w łatwy sposób przerzutować na dowolną kolekcję używając odpowiedniej metody (na przykład \texttt{.ToList()} albo \linebreak \texttt{.ToArray()}). Przykłady bardziej skomplikowanych zapytań można znaleźć w folderze \texttt{Thesis.Relinq.Tests} w plikach z rozszerzeniem \texttt{.cs} zawierających klasy testujące system, który stanowi załącznik do tej pracy.

\section{Drzewa wyrażeń \texttt{IQueryable}}
Wynikiem zapytania LINQ jest obiekt, który implementuje interfejs \texttt{IQueryable}. Poniższy fragment kodu pochodzi z biblioteki .NET i pokazuje sposób, w jaki \linebreak \texttt{IQueryable} rozszerza \texttt{IEnumerable}:

\begin{lstlisting}
public interface IQueryable : IEnumerable
{
    Type ElementType { get; }
    Expression Expression { get; }
    IQueryProvider Provider { get; }
}
\end{lstlisting}

Pierwsza właściwość zawiera oczywiście typ obiektów, których kolekcja jest wynikiem zapytania. Trzecia właściwość to instancja klasy, który implementuje interfejs \href{https://msdn.microsoft.com/pl-pl/library/system.linq.iqueryprovider(v=vs.110).aspx}{\texttt{IQueryProvider}}. Dostarczenie takiej implementacji jest zadaniem programisty, i o tym traktuje następna część rodziału. Natomiast przedmiotem tej sekcji jest właściwość druga, o tajemniczym typie \texttt{Expression}.

Prawdziwym ,,zapytaniem'' ukrytym pod interfejsem \texttt{IQueryable} jest obiekt \texttt{Expression}, który reprezentuje wejściowe zapytanie LINQ jako drzewo operatorów i metod, które zostały w tym zapytaniu użyte \cite{linq_queryable}. Po głębszej analizie kodu źródłowego biblioteki .NET okazuje się, że \texttt{IQueryable} jest tak naprawdę mechanizmem wykorzystującym metody typowe dla kolekcji do budowania drzewa rozbioru składniowego w postaci obiektu \texttt{Expression}, który (wraz z \texttt{ElementType}) jest wykorzystywany przez \texttt{Provider} do wykonania zapytania.

Może się wydawać, że mamy wszystko - przecież wystarczy zaimplementować \linebreak \texttt{IQueryProvider} w taki sposób, by tłumaczył drzewa \texttt{Expression} na zapytanie do języka, który nas interesuje. Okazuje się, że te drzewa mogą być problematycznym modelem do odwiedzania. Idąc śladem Fabiana Schmieda \cite{re-linq}, weźmy na warsztat pewne zapytanie i zobaczmy, w jaki sposób obiekt \texttt{Expression} jest budowany:

\begin{lstlisting}
var linqQuery3 = 
    from c in QueryFactory.CreateLinqQuery<Customer>()
    from o in c.Orders
    where o.OrderNumber == 1
    select new { c, o };
\end{lstlisting}

Na początku, takie zapytanie jest tłumaczone na równoważny ciąg wywołań metod (równie dobrze programista mógł napisać w kodzie to, co jest poniżej):

\begin{lstlisting}
QueryFactory.CreateLinqQuery<Customer>()
    .SelectMany(c => c.Orders, (c, o) => new {c, o})
    .Where(trans => trans.o.OrderNumber == 1)
    .Select(trans => new {trans.c, trans.o})
\end{lstlisting}

Kompilator tłumaczy powyższe wywołania metod na wywołania statycznych metod \texttt{IQueryable}, oraz opakowuje wyrażenia lambda w obiekty \linebreak \texttt{Expression.Lambda}, które są ich abstrakcyjną reprezentacją:

\begin{lstlisting}
Queryable.Select(
  Queryable.Where(
    Queryable.SelectMany(
      QueryFactory.CreateLinqQuery<Customer>(),
      Expression.Lambda(Expression.MakeMemberAccess(...)),
      Expression.Lambda(Expression.New(...))),
    Expression.Lambda(Expression.MakeBinary(...))),
  Expression.Lambda(Expression.New (...)))
\end{lstlisting}

Z tej reprezentacji korzystają obiekty \texttt{IQueryable}, które budują poszukiwany obiekt \texttt{Expression}, który wreszcie jest abstrakcyjną reprezentacją zapytania, które jest przekazywane do dostawcy LINQ w celu budowy zapytania:

\begin{lstlisting}
MethodCallExpression("Select",
  MethodCallExpression("Where",
    MethodCallExpression("SelectMany",
      CostantExpression(IQueryable<Customer>),
      UnaryExpression(...), UnaryExpression(...)),
    UnaryExpression(...)),
  UnaryExpression(...))
\end{lstlisting}

W tym miejscu warto zauważyć, że \texttt{Expression} jest oczywiście tylko klasą abstrakcyjną dla \href{https://msdn.microsoft.com/en-us/library/system.linq.expressions.expression(v=vs.110).aspx}{klas określających konkretne wyrażenia, które po niej dziedziczą}, takie jak \texttt{MethodCallExpression}, \texttt{UnaryExpression}, czy \texttt{BinaryExpression}.

Problemem z drzewami \texttt{Expression} jest fakt, że kolejność wykonywanych metod nie jest z góry określona - \textbf{jakakolwiek} metoda może nastąpić po \textbf{jakiejkolwiek} metodzie, przez co drzewa bardzo szybko stają się skomplikowane. Ponadto, jedna metoda może służyć w kilku kontekstach, np. \texttt{SelectMany} może służyć zarówno za część odpowiadającą za budowę podzapytania, jak również wybór dodatkowego źródła danych (następna tabela dla części \texttt{FROM} zapytania SQL-owego). Ponadto, dostawca LINQ musi przejść po wszystkich wyrażeniach lambda nawet na samą górę drzewa, aby znaleźć odpowiedni kontekst, o który chodziło użytkownikowi w zapytaniu. Stąd wniosek nasuwa się jeden - budowa dostawcy LINQ, który ma większe możliwości niż podstawowe operacje na pojedynczej tabeli, jest trudnym zadaniem, jeśli chciałoby się to zrobić na drzewach \texttt{Expression}.

Kończąć powyższe rozważania, Schmied zauważył że logika przetwarzania drzew \texttt{Expression} jest w każdym dostawcy LINQ niepotrzebnie duplikowana. W tym miejscu zadał pytanie: ,,\textit{Czy inteligentniejszym rozwiązaniem nie byłaby \textbf{jednokrotna} implementacja logiki przetwarzania drzew w sposób generyczny, z której mogą korzystać wszyscy dostawcy LINQ}''? To pytanie było motywacją do powstania biblioteki \href{https://github.com/re-motion/Relinq}{re-linq}. Autor pracy dyplomowej skorzystał z tej biblioteki, i o sposobie jej działania oraz użycia poświęcona została cała następna sekcja. 

\section{re-linq i obiekty \texttt{QueryModel}}
W sekcji traktującej o drzewach wyrażeń \texttt{IQueryable} dowiedziono, że ze względu na skomplikowaną strukturę tych drzew, budowa zapytania docelowego na podstawie tych drzew jest trudna. W związku z tym, alternatywnym rozwiązaniem jest wspomniana już biblioteka \href{https://github.com/re-motion/Relinq}{re-linq}, która tłumaczy drzewa wyrażeń \texttt{IQueryable} na drzewa rozbioru składniowego o wiele przystępniejsze do przeglądania, a dokładniej na obiekty \href{https://github.com/re-motion/Relinq/blob/82fdca6a4bfd942bb4a71dd20ab9c5af0aea0541/Core/QueryModel.cs}{\texttt{QueryModel}}, które o wiele bardziej przypominają oryginalne zapytanie LINQ. Te obiekty mają cztery właściwości:

\begin{itemize}
\item \texttt{SelectClause} - klauzula \href{https://github.com/re-motion/Relinq/blob/82fdca6a4bfd942bb4a71dd20ab9c5af0aea0541/Core/Clauses/SelectClause.cs}{\texttt{SelectClause}} określająca element, który jest wybierany w zapytaniu \texttt{select} z końca zapytania).
\item \texttt{MainFromClause} - klauzula \href{https://github.com/re-motion/Relinq/blob/82fdca6a4bfd942bb4a71dd20ab9c5af0aea0541/Core/Clauses/MainFromClause.cs}{\texttt{MainFromClause}} określająca główne źródło, z którego wybierane są informacje w zapytaniu (najbardziej zewnętrzny \texttt{from}).
\item \texttt{BodyClauses} - zbiór wyrażeń implementujących \href{https://github.com/re-motion/Relinq/blob/82fdca6a4bfd942bb4a71dd20ab9c5af0aea0541/Core/Clauses/IBodyClause.cs}{\texttt{IBodyClause}}, które definiują jakie dane są wybierane w zapytaniu i w jakiej kolejności (słowa kluczowe \texttt{where}, \texttt{orderby}, \texttt{join}, wewnętrzne \texttt{from}-y, które są przetrzymywane w klauzulach \texttt{AdditionalFrom}).
\item \texttt{ResultOperators} - zbiór wyrażeń dziedziczących po \href{https://github.com/re-motion/Relinq/blob/82fdca6a4bfd942bb4a71dd20ab9c5af0aea0541/Core/Clauses/ResultOperatorBase.cs}{\texttt{ResultOperatorBase}}, które wykonują logikę na zbiorze wynikowym (na przykład metody agregujące \texttt{Count()}, \texttt{Average()}, \texttt{Distinct()} i im podobne, operacje na zbiorach \texttt{Union()}, \texttt{Distinct()} i im podobne).
\end{itemize}

Biblioteka re-linq, poza przekształceniem obiektów \texttt{Expression} na \texttt{QueryModel}, pozwala również na znaczne uproszczenie implementacji \texttt{IQueryProvider}, udostępniając klasę abstrakcyjną \href{https://github.com/re-motion/Relinq/blob/82fdca6a4bfd942bb4a71dd20ab9c5af0aea0541/Core/QueryableBase.cs}{\texttt{QueryableBase}}, po której dziedziczy klasa budująca zapytanie docelowe. Klasa ta musi posiadać metodę \texttt{CreateQueryProvider}, która zwraca obiekt typu \texttt{IQueryProvider} wykorzystywany przez \texttt{IQueryable}. Takim obiektem może być oferany przez re-linq \href{https://github.com/re-motion/Relinq/blob/82fdca6a4bfd942bb4a71dd20ab9c5af0aea0541/Core/DefaultQueryProvider.cs}{\texttt{DefaultQueryProvider}}, który jest budowany z trzech argumentów: typu docelowego implementującego \texttt{IQueryable}, obiektu \texttt{QueryParser} dokonującego translacji drzewa \texttt{Expression} do obiektu \texttt{QueryModel} (istnieje możliwość napisania własnego tłumacza, ale autor pracy korzysta z domyślnego, który został dostarczony razem z biblioteką), oraz własnej implementacji interfejsu \href{https://github.com/re-motion/Relinq/blob/82fdca6a4bfd942bb4a71dd20ab9c5af0aea0541/Core/IQueryExecutor.cs}{\texttt{IQueryExecutor}} (patrz: \texttt{Thesis.Relinq/PsqlQueryable.cs}). Taka implementacja powinna posiadać trzy metody:

\begin{itemize}
\item \texttt{IEnumerable<T> ExecuteCollection<T>(QueryModel queryModel)},
\item \texttt{T ExecuteScalar<T>(QueryModel queryModel)},
\item \texttt{T ExecuteSingle<T>(QueryModel queryModel, bool defaultWhenEmpty)}.
\end{itemize}

Wybór wywoływanej przez \texttt{IQueryExecutor} metody zależy od oczekiwanego wyniku zapytania (cała kolekcja, skalar, pojedynczy element z kolekcji). W rezultacie, pisząc zapytanie LINQ, dostajemy obiekt w pełni implementujący \texttt{IQueryable}, na którym wywołanie metody wyciągającej wynik z bazy danych zwróci wynik jednej z powyższych trzech metod. Teraz jedyne, co nas dzieli od oczekiwanego rezultatu, jest ich implementacja, która przechodząc przez drzewo \texttt{QueryModel} buduje zapytanie, wykonuje je korzystając z zewnętrznej biblioteki łączącą się z bazą danych PostgreSQL, konwertuje wynik zapytania do oczekiwanego typu i go zwraca.

Sposobem budowy zapytania na podstawie obiektu \texttt{QueryModel} jest implementacja wzorca projektowego Odwiedzający (Visitor), którego zadaniem jest przejście przez wnętrze tego obiektu. Biblioteka re-linq oczywiście udostępnia bazowe klasy abstrakcyjne, które wystarczy przeciążyć w celu wykonania tego zadania, i o tym poświęcony został następny rozdział niniejszej pracy. Przy okazji warto jeszcze wspomnieć, że biblioteka re-linq jest na tyle potężnym narzędziem, że na jej użycie zdecydowali się nawet autorzy \href{https://github.com/nhibernate/nhibernate-core/blob/d82d1381fb6b427da91d357398502a7f4b482ccc/src/NHibernate/Linq/NhRelinqQueryParser.cs}{NHibernate} oraz \href{https://github.com/aspnet/EntityFramework/blob/f386095005e46ea3aa4d677e4439cdac113dbfb1/src/EFCore.Relational/Query/ExpressionVisitors/Internal/EqualityPredicateExpandingVisitor.cs}{Entity Framework 7}, które są najpopularniejszymi bibliotekami ORM w .NET.
\chapter{Proces budowy zapytania}
Znając sposób działania dostawców LINQ oraz budowę drzewa \texttt{QueryModel}, wystarczy opracować metodę przechodzenia przez te drzewa w celu budowy zapytania do bazy PostgreSQL. Punktem wejściowym dla projektu, który jest załącznikiem do niniejszej pracy, jest artykuł \cite{codeproject_nhibernate}, opisujący przykładową implementację dostawcy LINQ dla NHibernate.

\section{Implementacja \texttt{QueryModelVisitorBase}}
Korzystając z dotychczasowej wiedzy, następnym krokiem do wykonania jest implementacja metod odwiedzających nowe drzewo rozbioru składniowego. Również w tym przypadku biblioteka re-linq asystuje programistę w tym zadaniu, udostępniając klasę \href{https://github.com/re-motion/Relinq/blob/ab11f0997998a90e17e90dc58b215c3997d47311/Core/QueryModelVisitorBase.cs}{\texttt{QueryModelVisitorBase}}, która implementuje zbiór metod odwiedzających obiekt \texttt{QueryModel}. Klasa ta nie robi nic szczególnego, poza sprawdzeniem poprawności typu elementów przekazywanych w argumentach oraz ich zaakceptowaniu - typowa implementacja wzorca Odwiedzający (Visitor). Trzeba oczywiście napisać rozszerzenie tej klasy, która wykona dodatkową logikę na argumentach implementowanych metod, oraz wywoła bazową logikę z użyciem słowa kluczowego \texttt{base}.

Argumentami każdej z metod, które będą nadpisywane, są różne klauzule - skondensowane do postaci wygodnych obiektów - które występują w zapytaniu LINQ. Ich właściwościami są znane już obiekty \texttt{Expression}, jednak są one na tyle proste, że można łatwo się zająć ich odwiedzeniem, i o tym będzie traktować następna sekcja tego rozdziału. Na chwilę obecną załóżmy, że posiadamy generyczną metodę, która odwiedza każdy możliwy podtyp \texttt{Expression}, i na jego podstawie buduje fragment zapytania SQL-owego, który jest przekazywany do instancji klasy \texttt{QueryPartsAggregator}, służącej do łączenia takich fragmentów w pełne zapytanie SQL. Dokładna implementacja klasy, która jest tematem niniejszego podrozdziału, znajduje się w pliku \texttt{PsqlGeneratingQueryModelVisitor.cs}. Autor pracy zachęca czytelnika do zapoznawania się z nim w trakcie czytania następnych podsekcji.

\subsection{Metoda \texttt{VisitQueryModel}}
TODO.

\subsection{Metoda \texttt{VisitSelectClause}}
TODO.

\subsection{Metoda \texttt{VisitMainFromClause}}
TODO.

\subsection{Metoda \texttt{VisitWhereClause}}
TODO.

\subsection{Metoda \texttt{VisitOrderByClause}}
TODO.

\subsection{Metoda \texttt{VisitJoinClause}}
TODO.

\subsection{Metoda \texttt{VisitAdditionalFromClause}}
TODO.

\subsection{Metoda \texttt{VisitGroupJoinClause}}
TODO.

\subsection{Metoda \texttt{VisitResultOperator}}
TODO.

\section{Implementacja \texttt{RelinqExpressionVisitor}}
TODO.

\subsection{Metoda \texttt{VisitBinary}}
TODO.

\section{Czynności następujące budowę zapytania}
TODO.
\chapter{Testy jakości i wydajności}
Tutaj coś będzie jak złapię wenę.

\section{Sekcja}
Tutaj też.
\chapter{Podsumowanie}
W ramach niniejszej pracy dyplomowej powstał dostawca LINQ, który jest kompatybilny z bazą danych PostgreSQL i posiada duża liczbę funkcjonalności. Co więcej, dostawca ten jest łatwo rozszerzalny na inne dialekty SQL-a, ze względu na ich duże podobieństwo - prawdopodobnie funkcjonuje poprawnie dla prostych zapytań w innych bazach danych.

Zadanie to było rozwijające dla autora pracy nie tylko pod względem poznania możliwości i ograniczeń bazy PostgreSQL, ale również pozwoliło lepiej zrozumieć jak LINQ wygląda od wewnątrz w języku C\#. Trzeba pamiętać, że technologia LINQ umożliwia nie tylko wykonywanie zapytań do relacyjnych baz danych, ale również innych źródeł danych, a w szczególności baz danych typu NoSQL, i to jest jednym z tematów, z którymi autor pracy ma zamiar zapoznać się w przyszłości.

\begin{thebibliography}{1}
% I know this a bad way of doing things. BUT IT WORKS. :P

\bibitem{msdn_linq} Don Box, Anders Hejlsberg, \href{https://msdn.microsoft.com/en-us/library/bb308959.aspx}{LINQ: .NET Language-Integrated Query}, 2007.
\bibitem{linq_queryable} Matt Warren, \href{https://blogs.msdn.microsoft.com/mattwar/2007/07/30/linq-building-an-iqueryable-provider-part-i/}{LINQ: Building an IQueryable Provider}, 2007.
\bibitem{re-linq} Fabian Schmied, \href{https://www.re-motion.org/download/re-linq.pdf}{re-linq: A General Purpose LINQ Foundation}, 2009.
\bibitem{codeproject_nhibernate} Markus Giegl, \href{https://www.codeproject.com/Articles/42059/re-linq-ishing-the-Pain-Using-re-linq-to-Implement}{re-linq $\vert$ ishing the Pain: Using re-linq to Implement a Powerful LINQ Provider on the Example of NHibernate}, 2010.
\bibitem{postgres_doc} The PostgreSQL Global Development Group, \href{https://www.postgresql.org/docs/}{Documentation}, 1996-2017.
\end{thebibliography}

\renewcommand\appendixtocname{Dodatki}
\bookmarksetupnext{level=part}
\begin{appendices}
\addtocontents{toc}{\protect\setcounter{tocdepth}{1}}
\makeatletter
\addtocontents{toc}{
  \begingroup
  \let\protect\l@chapter\protect\l@section
  \let\protect\l@section\protect\l@subsection
}
\makeatother
  \chapter{Dodanie biblioteki do własnego projektu}
Biblioteka \texttt{Thesis.Relinq} została napisana w .NET Standard 1.4, co oznacza, że jest kompatybilna z projektami napisanymi zarówno w .NET Framework, jak również .NET Core. Niniejszy dodatek opisuje instrukcję dodania \texttt{Thesis.Relinq} do własnego projektu.

\section{.NET Framework}
todo.

\section{.NET Core}
todo.
\addtocontents{toc}{\endgroup}
\end{appendices}

\end{document}