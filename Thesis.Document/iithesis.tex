% Opcje klasy 'iithesis' opisane sa w komentarzach w pliku klasy. Za ich pomoca
% ustawia sie przede wszystkim jezyk i rodzaj (lic/inz/mgr) pracy, oraz czy na
% drugiej stronie pracy ma byc skladany wzor oswiadczenia o autorskim wykonaniu.
\documentclass[declaration,shortabstract,polish,inz]{iithesis}
\usepackage[utf8]{inputenc}
\usepackage{fancyhdr}
\usepackage[toc]{appendix}
\usepackage{bookmark}

\pagestyle{fancy}
\fancyhf{}
\fancyfoot[CO,CE]{\thepage}
\fancyhead[RE]{\leftmark}
\fancyhead[LO]{\rightmark}
\renewcommand{\chaptermark}[1]{\markboth{\thechapter\ #1}{}}
\renewcommand{\sectionmark}[1]{\markright{\thesection\ #1}}

\polishtitle{Implementacja języka zapytań oparta na\fmlinebreak abstrakcyjnych drzewach składniowych}
\englishtitle{Implementation of a query language\fmlinebreak based on abstract syntax trees}

\author{Damian Górski}
\advisor{dr Wiktor Zychla}
\date{30 czerwca 2017 r.} % Data zlozenia pracy
\transcriptnum{273212} % Numer indeksu
\advisorgen{dr Wiktora Zychli} % Nazwisko promotora w dopelniaczu

\polishabstract{Gdy projektujemy pewien system informatyczny, zazwyczaj musimy zmierzyć się z wyborem bazy danych, którą chcemy użyć w naszej aplikacji. Problem pojawia się w momencie, gdy pewna baza danych oferuje to, czego szukamy, ale potrzebujemy sposobu wybierania z niej informacji w wygodny dla nas sposób. W technologii .NET pozwala na to Language INtegrated Query (LINQ), który tłumaczy zapytanie w swoim języku na abstrakcyjne drzewo składniowe, po którym można przejść, implementując pewien zbiór obiektów odwiedzających (visitorów), w celu stworzenia zapytania w bazie danych naszego wyboru. Tematem tej pracy jest implementacja takich odwiedzających, którzy zbudują zapytanie do bazy PostgreSQL.}

\englishabstract{While projecting a computer system, we frequently have to cope with the task of choosing the database we want to use in our application. The problem is, some databases offer what we need, but we also need a more comfortable way to obtain information from it. The .NET framework allows us to achieve that with Language INtegrated Query (LINQ), which translates a query in its language into an abstract syntax tree, that we can traverse by implementing a set of visitors in order to create a query in our target database. The topic of this thesis is implementing such visitors that build a query to the PostgreSQL database.}

\begin{document}

\chapter{Miejscotrzymacz}
To są informacje, które chcę zawrzeć w tym dokumencie (podzielę je na rozdziały mniej więcej tak, jak są wypisane w punktach):
\begin{enumerate}
\item coś o IQueryable i w ogóle LINQ (w tym Expressions)
\item dlaczego oryginalne drzewa LINQ są trudne, słowo o re-linq (NHibernate używa re-linq!) i QueryModel
\item opis struktury projektu, przegląd najważniejszych komponentów, parafka o testach jednostkowych
\item instrukcja dodania projektu do projektu zewnętrznego (project.json)
\item www.ii.uni.wroc.pl/dla-studenta/prace-dyplomowe
\end{enumerate}

\chapter{Proces budowy zapytania}
Znając sposób działania dostawców LINQ oraz budowę drzewa \texttt{QueryModel}, wystarczy opracować metodę przechodzenia przez te drzewa w celu budowy zapytania do bazy PostgreSQL. Punktem wejściowym dla projektu, który jest załącznikiem do niniejszej pracy, jest artykuł \cite{codeproject_nhibernate}, opisujący przykładową implementację dostawcy LINQ dla NHibernate.

\section{Implementacja \texttt{QueryModelVisitorBase}}
Korzystając z dotychczasowej wiedzy, następnym krokiem do wykonania jest implementacja metod odwiedzających nowe drzewo rozbioru składniowego. Również w tym przypadku biblioteka re-linq asystuje programistę w tym zadaniu, udostępniając klasę \href{https://github.com/re-motion/Relinq/blob/ab11f0997998a90e17e90dc58b215c3997d47311/Core/QueryModelVisitorBase.cs}{\texttt{QueryModelVisitorBase}}, która implementuje zbiór metod odwiedzających obiekt \texttt{QueryModel}. Stawianym przed programistą zadaniem jest napisanie klasy dziedziczącej po \texttt{QueryModelVisitorBase}, która wykona dodatkową logikę na argumentach implementowanych metod, oraz wywoła bazową logikę z użyciem słowa kluczowego \texttt{base} w celu akceptowania odwiedzanych elementów.

Argumentami każdej z metod, które będą nadpisywane, są różne klauzule - skondensowane do postaci wygodnych obiektów - które występują w zapytaniu LINQ. Ich właściwościami są znane już obiekty \texttt{Expression}, jednak są one na tyle proste, że można łatwo się zająć ich odwiedzeniem, i o tym będzie traktować następna sekcja tego rozdziału. Na chwilę obecną załóżmy, że posiadamy generyczną metodę, która odwiedza każdy możliwy podtyp \texttt{Expression}, i na jego podstawie buduje fragment zapytania SQL-owego. Taki fragment jest przekazywany do instancji klasy \texttt{QueryPartsAggregator}, służącej do łączenia takich fragmentów w pełne zapytanie SQL. Dokładna implementacja klasy, która jest tematem niniejszego podrozdziału, znajduje się w pliku \texttt{PsqlGeneratingQueryModelVisitor.cs}. Autor pracy zachęca czytelnika do zapoznawania się z nim w trakcie czytania następnych podsekcji.

\subsection{Metoda \texttt{VisitQueryModel}}
Punkt wejściowy dla całego procesu odwiedzania całego zapytania. Dla zadanego \texttt{QueryModel}, wywołuje metody \texttt{VisitSelectClause}, \texttt{VisitMainFromClause} oraz zbiór metod odwiedzających po kolei każdy z elementów właściwości \linebreak \texttt{BodyClauses} i \texttt{ResultOperators}.

\subsection{Metoda \texttt{VisitSelectClause}}
Odwiedza klauzulę \texttt{SelectClause}, która definiuje właściwości obiektu, który zostanie zbudowany w wyniku zapytania (buduje część \texttt{SELECT} zapytania SQL-owego).

\subsection{Metoda \texttt{VisitMainFromClause}}
Odwiedza klauzulę \texttt{MainFromClause}, która definiuje źródło, na podstawie którego obiekt zostanie zbudowany w wyniku zapytania (dodaje pierwszą tabelę do części \texttt{FROM} w zapytaniu SQL-owym).

\subsection{Metoda \texttt{VisitWhereClause}}
W przypadku, gdy kolekcja \texttt{BodyClauses} zawiera klauzulę \texttt{WhereClause} (inaczej - zapytanie LINQ zawiera metodę \texttt{Where}), dodaje warunek, który wybrane dane muszą spełniać (dodaje element do części \texttt{WHERE} zapytania SQL-owego).

\subsection{Metoda \texttt{VisitOrderByClause}}
W przypadku, gdy kolekcja \texttt{BodyClauses} zawiera klauzulę \texttt{OrderByClause} (zapytanie LINQ zawiera metodę \texttt{OrderBy} lub \texttt{OrderByDescending}), dodaje porządek, według którego dane zostaną posortowane (dodaje element do części \texttt{ORDER BY} zapytania SQL-owego).

\subsection{Metoda \texttt{VisitJoinClause}}
W przypadku, gdy kolekcja \texttt{BodyClauses} zawiera klauzulę \texttt{JoinClause} (zapytanie LINQ zawiera metodę \texttt{Join}), dodaje złączenie wewnętrzne (ang. \textit{inner join}) do poprzedniego dodanego źródła danych w zapytaniu (dokleja \texttt{INNER JOIN [table]} do odpowiedniej części \texttt{FROM} zapytania SQL-owego, a dokładniej do tabeli, która jest łączona).

\subsection{Metoda \texttt{VisitAdditionalFromClause}}
W przypadku, gdy kolekcja \texttt{BodyClauses} zawiera klauzulę \texttt{FromClause} (zapytanie LINQ zawiera więcej niż jedną część \texttt{from}, która została zakumulowana razem z poprzednimi do większego obiektu metodą \texttt{SelectMany}), dodaje następne źródło, na podstawie którego obiekt zostanie zbudowany w wyniku zapytania (dodaje następną tabelę do części \texttt{FROM} zapytania SQL-owego po przecinku, co w rezultacie tworzy iloczyn kartezjański dwóch tabel, ang. \textit{cross join}).

\subsection{Metoda \texttt{VisitGroupJoinClause}}
W przypadku, gdy kolekcja \texttt{BodyClauses} zawiera klauzulę \texttt{GroupJoinClause} (zapytanie LINQ zawiera metodę \texttt{GroupJoin}), dodaje lewostronne złączenie zewnętrzne (ang. \textit{left join}) do poprzedniego dodanego źródła danych w zapytaniu (dokleja \texttt{LEFT JOIN [table]} do odpowiedniej części \texttt{FROM} zapytania SQL-owego, a dokładniej do tabeli, która jest łączona).

\subsection{Metoda \texttt{VisitResultOperator}}
W odróżnieniu od wszystkich powyższych klauzul, które implementują \linebreak \href{https://github.com/re-motion/Relinq/blob/82fdca6a4bfd942bb4a71dd20ab9c5af0aea0541/Core/Clauses/IBodyClause.cs}{\texttt{IBodyClause}}, re-linq niestety nie udostępnia wygodnego modelu odwiedzania dla obiektów \href{https://github.com/re-motion/Relinq/blob/82fdca6a4bfd942bb4a71dd20ab9c5af0aea0541/Core/Clauses/ResultOperatorBase.cs}{\texttt{ResultOperatorBase}}, w związku z tym ta metoda jest wywoływana dla każdego obiektu zawartego we właściwości \texttt{QueryModel.ResultOperators}.

Dostawca LINQ, który jest tematem niniejszej pracy, dzieli operatory wynikowe na pięć kategorii, zależnych od właściwego typu obiektu, który dziedziczy po \texttt{ResultOperatorBase} (każdy z nich nazywa się \texttt{SomeResultOperator}, dla prostoty każda nazwa została w poniższym spisie skrócona):

\begin{enumerate}[a)]
\item \texttt{Count}, \texttt{Average}, \texttt{Min}, \texttt{Max}, \texttt{Sum}, \texttt{Distinct} (operatory agregujące, które jako argument przyjmują zbiór wybranych danych i na ich podstawie zwraca pojedynczą wartość (w przypadku \texttt{Distinct} - unikatowe krotki)) - otacza wybraną część \texttt{SELECT} zapytania SQL-owego w odpowiadającą danemu operatorowi funkcję.
\item \texttt{Union, Intersect, Concat, Except} (operatory, które jako argumenty przyjmują zbiory danych i zwracają nowy zbiór) - sygnatury odpowiadających w języku C\# metod w jednym ze swoich argumentów mają zbiór, na którym ma zostać wykonana dana operacja. Ten zbiór jest oczywiście kolejnym drzewem \texttt{Expression}, które zostaje przetłumaczone na \texttt{QueryModel}, w związku z tym budowane zostaje podzapytanie, a zapytanie końcowe jest wynikiem złączenia zapytania głównego i podrzędnego. 
\item \texttt{Take}, \texttt{Skip} (operatory stronicowania) - dodaje do zapytania odpowiednią część odpowiedzialną za stronicowanie \texttt{(LIMIT X/OFFSET X)}.
\item \texttt{Any} (operator określający istnieje obiektu, który spełnia pewien warunek) - dolecowo użyty do wybierania obiektu na podstawie stwierdzenia, czy istnieje obiekt w wyniku innego zapytaniu, który spełnia podany na zewnątrz warunek. Poniższe zapytanie LINQ:

\begin{lstlisting}
QueryFactory.CreateLinqQuery<Customer>()
    .Where(c => QueryFactory.CreateLinqQuery<Order>()
    .Any(o => o.CustomerID == c.CustomerID));
\end{lstlisting}

zostaje tłumaczone na odpowiadające mu zapytanie w SQL-u:

\begin{lstlisting}
SELECT * FROM customers WHERE EXISTS 
  (SELECT * FROM orders WHERE
    (customers.CustomerID = orders.CustomerID));
\end{lstlisting}

% p ^ q <=> ~p ^ ~q
\item \texttt{All} (operator określający spełnienie pewnego warunku przez wszystkie obiekty w kolekcji) - dolecowo użyty do wybierania obiektu na podstawie stwierdzenia, czy wszystkie obiekty w wyniku innego zapytania spełniają podany na zewnątrz warunek. To stwierdzenie jest równoważne stwierdzeniu, że \textbf{nie istnieje} obiekt, który \textbf{nie spełnia} danego warunku (na zajęciach \textit{Logika dla informatyków} w IIUWr można dowiedzieć się, że $\forall x \phi \Leftrightarrow \neg \exists x \neg \phi$). Korzystając z tego faktu, poniższe zapytanie LINQ:

\begin{lstlisting}
QueryFactory.CreateLinqQuery<Customer>()
    .Where(c => QueryFactory.CreateLinqQuery<Order>()
    .All(o => o.CustomerID != c.CustomerID));
\end{lstlisting}

zostaje tłumaczone na następujące zapytanie w SQL-u:

\begin{lstlisting}
SELECT * FROM customers WHERE NOT EXISTS 
  (SELECT * FROM orders WHERE NOT
    (customers.CustomerID != orders.CustomerID));
\end{lstlisting}

\end{enumerate}

\section{Implementacja \texttt{RelinqExpressionVisitor}}
Implementacja klasy \texttt{QueryModelVisitorBase}, opisana w rozdziale 2.1, zajmuje się odwiedzaniem obiektu \texttt{QueryModel} oraz przetwarzaniem wygenerowanych części zapytania SQL-owego do postaci pary napisu przedstawiającego zapytanie oraz słownika z parametrami. W tym rozdziale opisana została implementacja udostępnianej przez re-linq klasy abstrakcyjnej \href{https://github.com/re-motion/Relinq/blob/82fdca6a4bfd942bb4a71dd20ab9c5af0aea0541/Core/Parsing/RelinqExpressionVisitor.cs}{\texttt{RelinqExpressionVisitor}}, która dziedziczy po .NET-owym \href{https://msdn.microsoft.com/en-us/library/system.linq.expressions.expressionvisitor(v=vs.110).aspx}{\texttt{ExpressionVisitor}}. Służy ona do generowania kluczowych części zapytania oraz parametrów, które dane zapytanie będzie wykorzystywać.

Argumentami każdej z metod, które będą nadpisywane, są obiekty dziedziczące po \texttt{Expression}. Na ich podstawie budowany jest napis w klasie \href{https://msdn.microsoft.com/pl-pl/library/system.text.stringbuilder(v=vs.110).aspx}{\texttt{StringBuilder}}, który po zakończeniu odwiedzania wyrażenia zostaje przekazany do omawianej już implementacji \texttt{QueryModelVisitorBase}.

\subsection{Metoda \texttt{VisitQuerySourceReference}}
Ta metoda odwiedza źródło danych, z którego wybrane zostaną dane. Rozpatrywane są dwa przypadki:

\begin{enumerate}[a)]
\item Źródło jest klauzulą \texttt{GroupJoinClause} - aby zrozumieć postać tej klauzuli, rozważmy najpierw następujące zapytanie:

\begin{lstlisting}
from c in QueryFactory.CreateLinqQuery<Customer>()
join o in QueryFactory.CreateLinqQuery<Order>()
on c.CustomerID equals o.CustomerID into orders
select new
{
    Customer = c.CustomerID,
    Orders = orders
};
\end{lstlisting}

W niniejszym zapytaniu obiekt \texttt{orders} jest kolekcją \texttt{IEnumerable<Order>} zamówień wykonanych przez konkretnych użytkowników, a zapytanie wynikowe tworzy obiekty anonimowe postaci numeru ID klienta i kolekcji zamówień, które dany klient zamówił. Przetłumaczenie tego zapytania do SQL jest trudne, ze względu na konieczność grupowania kolekcji i zwrócenia jej w postaci obiektu. Rozwiązanie, które wykorzystuje \href{https://msdn.microsoft.com/pl-pl/library/bb882643(v=vs.110).aspx}{LINQ to SQL} - oraz biblioteka autora pracy - jest dosyć sprytne: wykonywane jest złączenie zewnętrzne lewostronne tablicy grupującej z grupowaną, całość zostaje posortowana względem porównywanych kluczy, oraz dodawana jest nowa kolumna, która jest wynikiem podzapytania zliczającego obiekty w każdej grupie. Powyższe zapytanie LINQ-owe tłumaczone jest na:

\begin{lstlisting}
SELECT 
  customers.CustomerID AS CustomerID, [...], 
  (SELECT COUNT(*) FROM orders AS temp 
    WHERE temp.CustomerID = customers.CustomerID) 
  AS Orders.__GROUP_COUNT
FROM customers LEFT OUTER JOIN orders ON 
  customers.CustomerID = orders.CustomerID
ORDER BY customers.CustomerID, orders.CustomerID;
\end{lstlisting}

\texttt{GroupJoinClause} posiada właściwość \texttt{JoinClause}, z której wybierane są właściwości \texttt{[Outer/Inner]KeySelector}, na podstawie których doklejany zostaje powyższy kawałek zapytania umożliwiający grupowanie danych z poziomu LINQ.

\item Źródło nie jest klauzulą \texttt{GroupJoinClause} - w tym przypadku odwiedzana jest po prostu tablica w bazie danych. W zależności od tego, czy obecne wywołanie metody zostało wykonane przez metodę \texttt{VisitMember} lub nie, do zapytania doklejana jest nazwa tablicy lub ciąg postaci \texttt{[tablica].[kolumna1], [tablica].[kolumna2], ...}, który definiuje całą tablicę \texttt{tablica}.
\end{enumerate}

\subsection{Metoda \texttt{VisitSubQuery}}
Wyciąga z \texttt{SubQueryExpression} dodatkowy \texttt{QueryModel}, buduje na jego podstawie zapytanie i dodaje je do nadrzędnej klasy obsługującej budowę głównego zapytania. Jest to jedyna metoda, która nie generuje napisu w \texttt{StringBuilder}, a wykonuje logikę bezpośrednio na obiekcie odwiedzającym \texttt{QueryModel}. Takie zapytania są później łączone w całość za pomocą odpowiadających im operatorów wynikowych.

\subsection{Metoda \texttt{VisitBinary}}
Wyrażenia \texttt{BinaryExpression}, jak można się domyślić, mają jako właściwości wyrażenia \texttt{Left} i \texttt{Right} oraz operator łączący je. Odwiedza lewe wyrażenie, dokleja do wyniku napis odpowiadający operatorowi łączącemu, odwiedza prawe wyrażenie.

\subsection{Metoda \texttt{VisitConditional}}
Rozważmy następujące zapytanie LINQ:

\begin{lstlisting}
from e in QueryFactory.CreateLinqQuery<Entity>()
select new
{
    Result = (e.Property < 5 
        ? "less than five"
        : e.Property == 5 
            ? "five"
            : "more than five")
};
\end{lstlisting}

W ramach przypomnienia: operator \texttt{?} jest operatorem warunkowym, który ewaluuje wyrażenie boolowskie i zwraca wartość przed dwukropkiem dla prawdy, po dwukropku dla fałszu. W kontekście budowy zapytania, jest ono przetrzymywane w postaci \texttt{ConditionalExpression}, które zawiera właściwości \texttt{Test}, \texttt{IfTrue}, \texttt{IfFalse}. W szczególności, w \texttt{IfTrue} i \texttt{IfFalse} może być następne wyrażenie warunkowe. Metoda \texttt{VisitConditional} przechodzi po drzewie takich wyrażeń i tłumaczy je do SQL z użyciem funkcji \texttt{CASE}. Dla powyższego zapytania, odpowiadające mu zapytanie SQL wygląda następująco:

\begin{lstlisting}
SELECT
  CASE WHEN entities.Property < 5 THEN 'less than five'
       WHEN entities.Property = 5 THEN 'five'
       ELSE 'more than five'
  END AS "Result"
FROM entities;
\end{lstlisting}

\subsection{Metoda \texttt{VisitConstant}}
Aby zapobiec atakowi typu SQL injection, należy parametryzować zapytanie. Odwiedzając wartość stałą (jest nią np. napis, liczba, itp.), metoda tworzy nowy parametr w zapytaniu, nadaje mu nazwę i dokleja tą nazwę do zapytania.

\subsection{Metoda \texttt{VisitMember}}
Odwiedzana przy wyborze właściwości z modelu tabeli w bazie danych. Dokleja do zapytania (do części \texttt{SELECT}) napis \texttt{[table].[column]}, pozwalając na wybór pojedynczych kolumn w wyniku zapytania.

\subsection{Metoda \texttt{VisitMethodCall}}
Opakowuje metodę C\#-ową w odpowiadającą funkcję w zapytaniu SQL-owym. Przekazuje tej funkcji argumenty w sposób określony przez jej sygnaturę, po czym dopisuje dany fragment zapytania do bufora.

\subsection{Metoda \texttt{VisitNew}}
W przypadku, gdy zapytanie LINQ zwraca nowy obiekt anonimowy, ta metoda pozwala na przejście po wszystkich właściwościach nowego obiektu i ich odwiedzenie.

\subsection{Metoda \texttt{VisitUnary}}
Wykorzystywana w negacji wyrażenia boolowskiego lub do przekazania tego wyrażenia jako \texttt{MemberExpression}.

\section{Czynności wykonywane po budowie zapytania}
Potrafiąc zbudować dowolne zapytanie SQL-owe na podstawie zapytania LINQ, pozostaje już tylko kwestia wykonania go i zwrócenia wyniku w postaci obiektowej. Zapytanie może wykonać dowolny obiekt \texttt{DbConnection} kompatybilny z PostgreSQL (na przykład pochodzący z biblioteki \href{http://www.npgsql.org/}{Npgsql}), rozszerzony przez bibliotekę \href{https://github.com/StackExchange/Dapper}{Dapper} o metody \texttt{Query} i \texttt{Query<T>}, które umożliwiają wykonanie zapytania, którego wynik jest automatycznie rzutowany do postaci obiektowej.

Korzystając z Dappera, wykonywanie i mapowanie zapytań jest banalnie proste:

\begin{lstlisting}
var result = connection.Query<T>(statement, parameters);
\end{lstlisting}

W tym miejscu zmienna \texttt{result} jest typu \texttt{IEnumerable<T>}, gdzie \texttt{T} jest typem obiektu, który zwraca oryginalne zapytanie LINQ (jest ono typu \texttt{IQueryable<T>)}. Dapper wymaga, aby nazwy kolumn w relacjach mapowanych na obiekty były takie same, jak nazwy właściwości w klasie modelowej - stąd, podczas budowy zapytania, każda kolumna została przemianowana na nazwę odpowiadającej właściwości w klasie z użyciem słowa kluczowego \texttt{AS}.

Dapper niestety nie radzi sobie z typami anonimowymi ze względu na to, że nie istnieje publiczny konstruktor obiektów tego rodzaju. Obejściem tego problemu jest dostarczona przez autora pracy metoda rozszerzająca \texttt{QueryAnonymous<T>}. Dla typów zawierających tylko i wyłącznie właściwości proste (inaczej mówiąc: dla krotek, które zawierają tylko kolumny z bazy danych), wystarczy każdą kolumnę przerzutować do postaci tablicy i za pomocą statycznej metody \texttt{Activator.CreateInstance} utworzyć nowy obiekt anonimowy. Metoda ta pozwala również na grupowanie obiektów dla zapytań używających metody \texttt{GroupJoin} (patrz: 2.2.1a), która korzysta z dodatkowego pola w celu określenia liczby obiektów grupowanych.
\chapter{Testy jakości i wydajności}
Niniejsza sekcja jest krótkim raportem o możliwościach dostarczanych przez dostawcę LINQ implementowanego w ramach tej pracy dyplomowej oraz jego wydajności czasowej, porównanej z innymi rozwiązaniami dostępnymi na rynku.

\section{Dostępne funkcjonalności}
Autor niniejszej pracy podczas realizowania projektu używał techniki \textit{test-driven development}, tj. przed implementacją pewnej funkcjonalności napisał test, który ją pokrywa. Wszystkie testy są dostępne w katalogu \texttt{Thesis.Relinq.Tests}.

Biorąc pod uwagę wszystkie testy, które zostały napisane, zaimplementowanymi funkcjonalnościami są:

\begin{itemize}
\setlength\itemsep{0em}
\item Połączenie się z bazą danych.
\item Obsługa podstawowych operatorów binarnych (dodawanie, logiczny OR, itp.).
\item Obsługa funkcji agregujących (Sum, Average, Min, Max, Count, Distinct).
\item Obsługa funkcji na napisach.
\item Budowa zapytania zwracającego kolekcję obiektów modelowych.
\item Budowa zapytania zwracającego kolekcję obiektów anonimowych.
\item Możliwość dodania jednego lub więcej wyrażenia warunkowego do zapytania.
\item Możliwość dodania wyniku rozwinięcia drzewa warunkowego do zapytania \linebreak (\texttt{CASE WHEN ... THEN ... END}).
\item Obsługa zapytań z wyrażeniem \texttt{EXISTS}.
\item Możliwość dodania porządkowania wyniku zapytania.
\item Możliwość stronicowania wyniku zapytania.
\item Możliwość łączenia kilku źródeł danych za pomocą złączeń wewnętrznych \linebreak (\textit{inner join}), zewnętrznych obustronnych (\textit{cross join}), zewnętrznych jednostronnych (\textit{outer left/right join}).
\item Możliwość wykonania C\#-owego \texttt{GroupJoin} na tabelach w PostgreSQL,
\item Obsługa operatorów działających na zbiorach (\texttt{UNION, UNION ALL, \linebreak INTERSECT, EXCEPT}).
\item Parametryzacja zapytań w celu zwiększenia bezpieczeństwa.
\end{itemize}

Funkcjonalnościami, które są ważne, lecz niestety nie zostały zaimplementowane z powodu ich trudności, są:

\begin{itemize}
\item Grupowanie z funkcją agregującą z użyciem \texttt{GROUP BY} - C\#-owa metoda \texttt{GroupBy} działa w inny sposób, niż funkcja \texttt{GROUP BY} w zapytaniach SQL-owych. W odróżnieniu od SQL-a, grupowanie w C\# może przebiegać nie tylko po funkcjach agregujących, takich jak \texttt{COUNT()} lub \texttt{AVG()}, ale również po właściwościach, czy nawet całych obiektach.
\item C\#-owe metody \href{https://msdn.microsoft.com/pl-pl/library/bb534804(v=vs.110).aspx}{\texttt{TakeWhile}} i \href{https://www.google.pl/url?sa=t&rct=j&q=&esrc=s&source=web&cd=1&cad=rja&uact=8&ved=0ahUKEwj43ai66LPUAhVCVhQKHSm4BfkQFggnMAA&url=https%3A%2F%2Fmsdn.microsoft.com%2Fpl-pl%2Flibrary%2Fbb549075(v%3Dvs.110).aspx&usg=AFQjCNHEbD9WVhCdEeCDPQw0CIt845i9Kg&sig2=41OZNQAxqA1IYG9QzuzK1g}{\texttt{SkipWhile}} - ze względu na brak możliwości napisania zapytania SQL-owego odpowiadającemu tym metodom bez wykorzystania procedur składowanych (ang. \textit{stored procedures}), implementacja takiej funkcjonalności jest prawdopodobnie niemożliwa; w tymi metodami nie radzi sobie żaden ORM (nawet Entity Framework), który jest znany autorowi pracy.
\item Automatyczne wyznaczanie kontekstu z bazy danych - istnieje konieczność ręcznej budowy klas i ich właściwości, które mają odpowiadać tabelom i ich kolumnom w bazie danych, do której pisane jest zapytanie. Dobrym rozwiązaniem na nadanie swobody nazwy takiej klasy i jej właściwości byłoby przydzielanie im własnych atrybutów, tak jak realizowane jest to w bibliotece LINQ to DB, której przyjrzymy się w ramach testów wydajnościowych w następnej sekcji.
\end{itemize}

\section{Wydajność a inne rozwiązania}
Z poprzedniej sekcji wynika, że przeciętny student informatyki w ciągu czterech miesięcy jest w stanie napisać w ramach pracy dyplomowej dostawcę LINQ z nietrywialnymi funkcjonalnościami. Czas sprawdzić wydajność takiego dostawcy, porównując go z komercyjnym LinqConnect firmy DevArt, dołączonym do próbnej wersji biblioteki \href{https://www.devart.com/dotconnect/postgresql/}{dotConnect for PostgreSQL 7.9 Professional}, oraz open-sourcowym \href{https://github.com/linq2db/linq2db}{LINQ to DB}, które jest rozwijane od kilku lat. W celu sprawdzenia wydajności tych trzech dostawców LINQ dla PostgreSQL, użyta została biblioteka do testowania wydajności w .NET o nazwie \href{https://github.com/dotnet/BenchmarkDotNet}{BenchmarkDotNet}.

W ramach sprawiedliwego pomiaru, każda z trzech bibliotek otrzymała identyczne zapytania do wykonania na tej samej bazie danych. Poniższa tabela przedstawia czasy wykonania pojedynczego zapytania określonego rodzaju (nazwy testów powinny mówić same za siebie):


\chapter{Podsumowanie}
W ramach niniejszej pracy dyplomowej powstał dostawca LINQ, który jest kompatybilny z bazą danych PostgreSQL i posiada duża liczbę funkcjonalności. Co więcej, dostawca ten jest łatwo rozszerzalny na inne dialekty SQL-a, ze względu na ich duże podobieństwo – prawdopodobnie funkcjonuje poprawnie dla prostych zapytań w innych bazach danych.

Zadanie to było rozwijające dla autora pracy nie tylko pod względem poznania możliwości i ograniczeń bazy PostgreSQL, ale również pozwoliło lepiej zrozumieć jak LINQ wygląda od wewnątrz w języku C\#. Trzeba pamiętać, że technologia LINQ umożliwia nie tylko wykonywanie zapytań do relacyjnych baz danych, ale również innych źródeł danych, a w szczególności baz danych typu NoSQL, i to jest jednym z tematów, z którymi autor pracy ma zamiar zapoznać się w przyszłości.

%\begin{appendices}
%\chapter{Instrukcja obsługi}
Tutaj coś będzie jak złapię wenę.
%end{appendices}

\bookmarksetupnext{level=part}
\begin{thebibliography}{1}
% http://www.codeproject.com/KB/linq/relinqish_the_pain.aspx
% https://www.re-motion.org/blogs/mix/archive/2009/09/02/how-to-write-a-linq-provider-the-simple-way-again.aspx
%\bibitem{example} \ldots
\end{thebibliography}

\renewcommand\appendixtocname{Dodatki}
\bookmarksetupnext{level=part}
\begin{appendices}
\addtocontents{toc}{\protect\setcounter{tocdepth}{1}}
\makeatletter
\addtocontents{toc}{
  \begingroup
  \let\protect\l@chapter\protect\l@section
  \let\protect\l@section\protect\l@subsection
}
\makeatother
  \chapter{Instrukcja obsługi}
Tutaj coś będzie jak złapię wenę.
\addtocontents{toc}{\endgroup}
\end{appendices}

\end{document}