\chapter{Testy jakości i wydajności}
Tutaj coś jeszcze dzisiaj napiszę.

\section{Możliwe funkcjonalności}
Autor niniejszej pracy podczas realizowania projektu używał techniki \textit{test-driven development}, tj. przed implementacją pewnej funkcjonalności napisał test, który ją pokrywa. Wszystkie testy są dostępne w katalogu \texttt{Thesis.Relinq.Tests}.

\section{Wydajność a inne rozwiązania}
Z poprzedniej sekcji wynika, że przeciętny student informatyki w ciągu czterech miesięcy jest w stanie napisać w ramach pracy dyplomowej dostawcę LINQ z nietrywialnymi funkcjonalnościami. Czas sprawdzić wydajność takiego dostawcy, porównując go z komercyjnym LinqConnect firmy DevArt, dołączonym do próbnej wersji biblioteki \href{https://www.devart.com/dotconnect/postgresql/}{dotConnect for PostgreSQL 7.9 Professional}, oraz open-sourcowym \href{https://github.com/linq2db/linq2db}{LINQ to DB}, które jest rozwijane od kilku lat. W celu sprawdzenia wydajności tych trzech dostawców LINQ dla PostgreSQL, użyta została biblioteka do testowania wydajności w .NET o nazwie \href{https://github.com/dotnet/BenchmarkDotNet}{BenchmarkDotNet}.