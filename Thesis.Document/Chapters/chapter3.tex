\chapter{Testy jakości i wydajności}
Niniejsza sekcja jest krótkim raportem o możliwościach dostarczanych przez dostawcę LINQ implementowanego w ramach tej pracy dyplomowej oraz jego wydajności czasowej, porównanej z innymi rozwiązaniami dostępnymi na rynku.

\section{Dostępne funkcjonalności}
Autor niniejszej pracy podczas realizowania projektu używał techniki \textit{test-driven development}, tj. przed implementacją pewnej funkcjonalności napisał test, który ją pokrywa. Wszystkie testy są dostępne w katalogu \texttt{Thesis.Relinq.Tests}.

Biorąc pod uwagę wszystkie testy, które zostały napisane, zaimplementowanymi funkcjonalnościami są:

\begin{itemize}
\setlength\itemsep{0em}
\item Połączenie się z bazą danych.
\item Obsługa podstawowych operatorów binarnych (dodawanie, logiczny OR, itp.).
\item Obsługa funkcji agregujących (Sum, Average, Min, Max, Count, Distinct).
\item Obsługa funkcji na napisach.
\item Budowa zapytania zwracającego kolekcję obiektów modelowych.
\item Budowa zapytania zwracającego kolekcję obiektów anonimowych.
\item Możliwość dodania jednego lub więcej wyrażenia warunkowego do zapytania.
\item Możliwość dodania wyniku rozwinięcia drzewa warunkowego do zapytania \linebreak (\texttt{CASE WHEN ... THEN ... END}).
\item Obsługa zapytań z wyrażeniem \texttt{EXISTS}.
\item Możliwość dodania porządkowania wyniku zapytania.
\item Możliwość stronicowania wyniku zapytania.
\item Możliwość łączenia kilku źródeł danych za pomocą złączeń wewnętrznych \linebreak (\textit{inner join}), zewnętrznych obustronnych (\textit{cross join}), zewnętrznych jednostronnych (\textit{outer left/right join}).
\item Możliwość wykonania C\#-owego \texttt{GroupJoin} na tabelach w PostgreSQL,
\item Obsługa operatorów działających na zbiorach (\texttt{UNION, UNION ALL, \linebreak INTERSECT, EXCEPT}).
\item Parametryzacja zapytań w celu zwiększenia bezpieczeństwa.
\item Możliwość dowolnego mianowania nazw tabel i kolumn w bazie danych z poziomu klas modelowych.
\end{itemize}

Funkcjonalnościami, które są ważne, lecz niestety nie zostały zaimplementowane z powodu ich trudności, są:

\begin{itemize}
\item Grupowanie z funkcją agregującą z użyciem \texttt{GROUP BY} – C\#-owa metoda \texttt{GroupBy} działa w inny sposób, niż funkcja \texttt{GROUP BY} w zapytaniach SQL-owych. W odróżnieniu od SQL-a, grupowanie w C\# może przebiegać nie tylko po funkcjach agregujących, takich jak \texttt{COUNT()} lub \texttt{AVG()}, ale również po właściwościach, czy nawet całych obiektach.
\item C\#-owe metody \href{https://msdn.microsoft.com/pl-pl/library/bb534804(v=vs.110).aspx}{\texttt{TakeWhile}} i \href{https://www.google.pl/url?sa=t&rct=j&q=&esrc=s&source=web&cd=1&cad=rja&uact=8&ved=0ahUKEwj43ai66LPUAhVCVhQKHSm4BfkQFggnMAA&url=https%3A%2F%2Fmsdn.microsoft.com%2Fpl-pl%2Flibrary%2Fbb549075(v%3Dvs.110).aspx&usg=AFQjCNHEbD9WVhCdEeCDPQw0CIt845i9Kg&sig2=41OZNQAxqA1IYG9QzuzK1g}{\texttt{SkipWhile}} – ze względu na brak możliwości napisania zapytania SQL-owego odpowiadającemu tym metodom bez wykorzystania procedur składowanych (ang. \textit{stored procedures}), implementacja takiej funkcjonalności jest prawdopodobnie niemożliwa; w tymi metodami nie radzi sobie żaden ORM (nawet Entity Framework), który jest znany autorowi pracy.
\item Automatyczne wyznaczanie kontekstu z bazy danych – istnieje konieczność ręcznej budowy klas i ich właściwości, które mają odpowiadać tabelom i ich kolumnom w bazie danych, do której pisane jest zapytanie. Z drugiej strony, obecne rozwiązanie polegające na ręcznym pisaniu klas opisujących tabele w bazie danych, zezwala na dostarczanie nietrywialnych metod dla encji, które są wynikiem zapytania.
\end{itemize}

\newpage

\section{Wydajność a inne rozwiązania}
Z poprzedniej sekcji wynika, że student informatyki w ciągu czterech miesięcy jest w stanie napisać w ramach pracy dyplomowej dostawcę LINQ z nietrywialnymi funkcjonalnościami. Czas sprawdzić wydajność takiego dostawcy, porównując go z dwoma popularnymi dostawcami LINQ: komercyjnym LinqConnect firmy DevArt, dołączonym do próbnej wersji biblioteki \href{https://www.devart.com/dotconnect/postgresql/}{dotConnect for PostgreSQL 7.9 Professional}, oraz open-sourcowym \href{https://github.com/linq2db/linq2db}{LINQ to DB}, który jest rozwijany od kilku lat. W celu sprawdzenia wydajności tych trzech dostawców LINQ dla PostgreSQL, użyta została biblioteka do testowania wydajności w .NET o nazwie \href{https://github.com/dotnet/BenchmarkDotNet}{BenchmarkDotNet}.

W ramach sprawiedliwego pomiaru, każda z trzech bibliotek wykonuje identyczne zapytania do tej samej bazy danych, postawionej na tej samej maszynie, z której wykonywany jest kod kliencki (średniej klasy laptop). Każdy z testów jest załącznikiem do niniejszej pracy, i można je znaleźć w plikach \texttt{(...)Benchmark.cs} w katalogu \texttt{Thesis.Relinq.Benchmarks}. Poniższa tabela przedstawia czasy wykonania pojedynczego zapytania w każdym z testów:

\vspace{0.2in}
\begin{table}[h]
\centering
\hspace*{-0.075in}\begin{tabular}{l|ccc}
Nazwa metody                           & \textbf{LinqConnect} & \textbf{linq2db} & \textbf{ThesisRelinq} \\
\hline
select\_all                            & 0,712 ms             & 1,249 ms         & 1,450 ms              \\
select\_anonymous\_type                & 0,537 ms             & 0,540 ms         & 0,959 ms              \\
select\_with\_where                    & 0,465 ms             & 0,313 ms         & 0,697 ms              \\
select\_with\_multiconditional\_where  & 0,550 ms             & 0,487 ms         & 0,871 ms              \\
select\_with\_multiple\_wheres         & 0,579 ms             & 0,496 ms         & 0,907 ms              \\
select\_with\_case                     & 0,516 ms             & N/A              & 0,697 ms              \\
select\_with\_orderings\_joined        & 0,580 ms             & 0,607 ms         & 0,992 ms              \\
select\_with\_orderings\_split         & 0,592 ms             & 0,606 ms         & 0,984 ms              \\
select\_with\_take\_while              & N/A                  & N/A              & N/A                   \\
select\_with\_cross\_join              & 2,819 ms             & 2,810 ms         & 5,159 ms              \\
select\_with\_inner\_join              & 2,906 ms             & 2,841 ms         & 4,829 ms              \\
select\_with\_group\_join              & 61,257 ms            & 70,886 ms        & 136,525 ms            \\
select\_with\_outer\_join              & 2,938 ms             & 3,152 ms         & 5,090 ms              \\
select\_with\_paging                   & 0,513 ms             & 0,453 ms         & 0,878 ms              \\
select\_where\_any\_matches\_condition & 1,661 ms             & 1,795 ms         & 2,638 ms              \\
select\_where\_all\_match\_condition   & 1,306 ms             & 0,819 ms         & 1,623 ms              \\
select\_with\_union                    & 0,799 ms             & N/A              & 1,180 ms              \\
select\_with\_concat\_as\_union\_all   & 0,694 ms             & N/A              & 0,983 ms              \\
select\_with\_intersect                & 0,897 ms             & 0,622 ms         & 1,353 ms              \\
select\_with\_except                   & 0,892 ms             & 0,658 ms         & 1,345 ms  
\end{tabular}
\end{table}

\pagebreak

Zakładając, że pierwsze dwie biblioteki działają w optymalnym czasie, można zauważyć, że dostawca LINQ autora pracy działa około dwa razy wolniej od rozwiązań optymalnych. Nie jest to najgorszy wynik, ale w dalszym ciągu może ulec poprawie – dużo czasu poświęcane jest na budowę obiektów będących wynikiem zapytania, korzystając z mechanizmu refleksji w celu dynamicznego ustawiania właściwości nowopowstałych instancji klas oraz metody \texttt{Activator.CreateInstance}. Wzorcową alternatywą dla tego rozwiązania może być dynamiczne generowanie kodu za pomocą klasy \href{https://msdn.microsoft.com/pl-pl/library/system.reflection.emit.ilgenerator(v=vs.110).aspx}{\texttt{ILGenerator}}, która w trakcie kompilacji wygenerowałaby i wyemitowała do kompilatora fragment kodu tworzący w efektywny sposób nowe instancje obiektu będącego wynikiem zapytania \cite{il_generator}.