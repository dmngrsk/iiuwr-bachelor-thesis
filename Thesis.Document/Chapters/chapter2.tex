\chapter{Proces budowy zapytania}
Znając sposób działania dostawców LINQ oraz budowę drzewa \texttt{QueryModel}, wystarczy opracować metodę przechodzenia przez te drzewa w celu budowy zapytania do bazy PostgreSQL. Punktem wejściowym dla projektu, który jest załącznikiem do niniejszej pracy, jest artykuł \cite{codeproject_nhibernate}, opisujący przykładową implementację dostawcy LINQ dla NHibernate.

\section{Implementacja \texttt{QueryModelVisitorBase}}
Korzystając z dotychczasowej wiedzy, następnym krokiem do wykonania jest implementacja metod odwiedzających nowe drzewo rozbioru składniowego. Również w tym przypadku biblioteka re-linq asystuje programistę w tym zadaniu, udostępniając klasę \href{https://github.com/re-motion/Relinq/blob/ab11f0997998a90e17e90dc58b215c3997d47311/Core/QueryModelVisitorBase.cs}{\texttt{QueryModelVisitorBase}}, która implementuje zbiór metod odwiedzających obiekt \texttt{QueryModel}. Stawianym przed programistą zadaniem jest napisanie klasy dziedziczącej po \texttt{QueryModelVisitorBase}, która wykona dodatkową logikę na argumentach implementowanych metod, oraz wywoła bazową logikę z użyciem słowa kluczowego \texttt{base} w celu akceptowania odwiedzanych elementów.

Argumentami każdej z metod, które będą nadpisywane, są różne klauzule - skondensowane do postaci wygodnych obiektów - które występują w zapytaniu LINQ. Ich właściwościami są znane już obiekty \texttt{Expression}, jednak są one na tyle proste, że można łatwo się zająć ich odwiedzeniem, i o tym będzie traktować następna sekcja tego rozdziału. Na chwilę obecną załóżmy, że posiadamy generyczną metodę, która odwiedza każdy możliwy podtyp \texttt{Expression}, i na jego podstawie buduje fragment zapytania SQL-owego. Taki fragment jest przekazywany do instancji klasy \texttt{QueryPartsAggregator}, służącej do łączenia takich fragmentów w pełne zapytanie SQL. Dokładna implementacja klasy, która jest tematem niniejszego podrozdziału, znajduje się w pliku \texttt{PsqlGeneratingQueryModelVisitor.cs}. Autor pracy zachęca czytelnika do zapoznawania się z nim w trakcie czytania następnych podsekcji.

\subsection{Metoda \texttt{VisitQueryModel}}
Punkt wejściowy dla całego procesu odwiedzania całego zapytania. Dla zadanego \texttt{QueryModel}, wywołuje metody \texttt{VisitSelectClause}, \texttt{VisitMainFromClause} oraz zbiór metod odwiedzających po kolei każdy z elementów właściwości \linebreak \texttt{BodyClauses} i \texttt{ResultOperators}.

\subsection{Metoda \texttt{VisitSelectClause}}
Odwiedza klauzulę \texttt{SelectClause}, która definiuje właściwości obiektu, który zostanie zbudowany w wyniku zapytania (buduje część \texttt{SELECT} zapytania SQL-owego).

\subsection{Metoda \texttt{VisitMainFromClause}}
Odwiedza klauzulę \texttt{MainFromClause}, która definiuje źródło, na podstawie którego obiekt zostanie zbudowany w wyniku zapytania (dodaje pierwszą tabelę do części \texttt{FROM} w zapytaniu SQL-owym).

\subsection{Metoda \texttt{VisitWhereClause}}
W przypadku, gdy kolekcja \texttt{BodyClauses} zawiera klauzulę \texttt{WhereClause} (inaczej - zapytanie LINQ zawiera metodę \texttt{Where}), dodaje warunek, który wybrane dane muszą spełniać (dodaje element do części \texttt{WHERE} zapytania SQL-owego).

\subsection{Metoda \texttt{VisitOrderByClause}}
W przypadku, gdy kolekcja \texttt{BodyClauses} zawiera klauzulę \texttt{OrderByClause} (zapytanie LINQ zawiera metodę \texttt{OrderBy} lub \texttt{OrderByDescending}), dodaje porządek, według którego dane zostaną posortowane (dodaje element do części \texttt{ORDER BY} zapytania SQL-owego).

\subsection{Metoda \texttt{VisitJoinClause}}
W przypadku, gdy kolekcja \texttt{BodyClauses} zawiera klauzulę \texttt{JoinClause} (zapytanie LINQ zawiera metodę \texttt{Join}), dodaje złączenie wewnętrzne (ang. \textit{inner join}) do poprzedniego dodanego źródła danych w zapytaniu (dokleja \texttt{INNER JOIN [table]} do odpowiedniej części \texttt{FROM} zapytania SQL-owego, a dokładniej do tabeli, która jest łączona).

\subsection{Metoda \texttt{VisitAdditionalFromClause}}
W przypadku, gdy kolekcja \texttt{BodyClauses} zawiera klauzulę \texttt{FromClause} (zapytanie LINQ zawiera więcej niż jedną część \texttt{from}, która została zakumulowana razem z poprzednimi do większego obiektu metodą \texttt{SelectMany}), dodaje następne źródło, na podstawie którego obiekt zostanie zbudowany w wyniku zapytania (dodaje następną tabelę do części \texttt{FROM} zapytania SQL-owego po przecinku, co w rezultacie tworzy iloczyn kartezjański dwóch tabel, ang. \textit{cross join}).

\subsection{Metoda \texttt{VisitGroupJoinClause}}
W przypadku, gdy kolekcja \texttt{BodyClauses} zawiera klauzulę \texttt{GroupJoinClause} (zapytanie LINQ zawiera metodę \texttt{GroupJoin}), dodaje lewostronne złączenie zewnętrzne (ang. \textit{left join}) do poprzedniego dodanego źródła danych w zapytaniu (dokleja \texttt{LEFT JOIN [table]} do odpowiedniej części \texttt{FROM} zapytania SQL-owego, a dokładniej do tabeli, która jest łączona).

\subsection{Metoda \texttt{VisitResultOperator}}
W odróżnieniu od wszystkich powyższych klauzul, które implementują \linebreak \href{https://github.com/re-motion/Relinq/blob/82fdca6a4bfd942bb4a71dd20ab9c5af0aea0541/Core/Clauses/IBodyClause.cs}{\texttt{IBodyClause}}, re-linq niestety nie udostępnia wygodnego modelu odwiedzania dla obiektów \href{https://github.com/re-motion/Relinq/blob/82fdca6a4bfd942bb4a71dd20ab9c5af0aea0541/Core/Clauses/ResultOperatorBase.cs}{\texttt{ResultOperatorBase}}, w związku z tym ta metoda jest wywoływana dla każdego obiektu zawartego we właściwości \texttt{QueryModel.ResultOperators}.

Dostawca LINQ, który jest tematem niniejszej pracy, dzieli operatory wynikowe na pięć kategorii, zależnych od właściwego typu obiektu, który dziedziczy po \texttt{ResultOperatorBase} (każdy z nich nazywa się \texttt{SomeResultOperator}, dla prostoty każda nazwa została w poniższym spisie skrócona):

\begin{enumerate}[a)]
\item \texttt{Count}, \texttt{Average}, \texttt{Min}, \texttt{Max}, \texttt{Sum}, \texttt{Distinct} (operatory agregujące, które jako argument przyjmują zbiór wybranych danych i na ich podstawie zwraca pojedynczą wartość (w przypadku \texttt{Distinct} - unikatowe krotki)) - otacza wybraną część \texttt{SELECT} zapytania SQL-owego w odpowiadającą danemu operatorowi funkcję.
\item \texttt{Union, Intersect, Concat, Except} (operatory, które jako argumenty przyjmują zbiory danych i zwracają nowy zbiór) - sygnatury odpowiadających w języku C\# metod w jednym ze swoich argumentów mają zbiór, na którym ma zostać wykonana dana operacja. Ten zbiór jest oczywiście kolejnym drzewem \texttt{Expression}, które zostaje przetłumaczone na \texttt{QueryModel}, w związku z tym budowane zostaje podzapytanie, a zapytanie końcowe jest wynikiem złączenia zapytania głównego i podrzędnego. 
\item \texttt{Take}, \texttt{Skip} (operatory stronicowania) - dodaje do zapytania odpowiednią część odpowiedzialną za stronicowanie \texttt{(LIMIT X/OFFSET X)}.
\item \texttt{Any} (operator określający istnieje obiektu, który spełnia pewien warunek) - dolecowo użyty do wybierania obiektu na podstawie stwierdzenia, czy istnieje obiekt w wyniku innego zapytaniu, który spełnia podany na zewnątrz warunek. Poniższe zapytanie LINQ:

\begin{lstlisting}
QueryFactory.CreateLinqQuery<Customer>()
    .Where(c => QueryFactory.CreateLinqQuery<Order>()
    .Any(o => o.CustomerID == c.CustomerID));
\end{lstlisting}

zostaje tłumaczone na odpowiadające mu zapytanie w SQL-u:

\begin{lstlisting}
SELECT * FROM customers WHERE EXISTS 
  (SELECT * FROM orders WHERE
    (customers.CustomerID = orders.CustomerID));
\end{lstlisting}

% p ^ q <=> ~p ^ ~q
\item \texttt{All} (operator określający spełnienie pewnego warunku przez wszystkie obiekty w kolekcji) - dolecowo użyty do wybierania obiektu na podstawie stwierdzenia, czy wszystkie obiekty w wyniku innego zapytania spełniają podany na zewnątrz warunek. To stwierdzenie jest równoważne stwierdzeniu, że \textbf{nie istnieje} obiekt, który \textbf{nie spełnia} danego warunku (na zajęciach \textit{Logika dla informatyków} w IIUWr można dowiedzieć się, że $\forall x \phi \Leftrightarrow \neg \exists x \neg \phi$). Korzystając z tego faktu, poniższe zapytanie LINQ:

\begin{lstlisting}
QueryFactory.CreateLinqQuery<Customer>()
    .Where(c => QueryFactory.CreateLinqQuery<Order>()
    .All(o => o.CustomerID != c.CustomerID));
\end{lstlisting}

zostaje tłumaczone na następujące zapytanie w SQL-u:

\begin{lstlisting}
SELECT * FROM customers WHERE NOT EXISTS 
  (SELECT * FROM orders WHERE NOT
    (customers.CustomerID != orders.CustomerID));
\end{lstlisting}

\end{enumerate}

\section{Implementacja \texttt{RelinqExpressionVisitor}}
Implementacja klasy \texttt{QueryModelVisitorBase}, opisana w rozdziale 2.1, zajmuje się odwiedzaniem obiektu \texttt{QueryModel} oraz przetwarzaniem wygenerowanych części zapytania SQL-owego do postaci pary napisu przedstawiającego zapytanie oraz słownika z parametrami. W tym rozdziale opisana została implementacja udostępnianej przez re-linq klasy abstrakcyjnej \href{https://github.com/re-motion/Relinq/blob/82fdca6a4bfd942bb4a71dd20ab9c5af0aea0541/Core/Parsing/RelinqExpressionVisitor.cs}{\texttt{RelinqExpressionVisitor}}, która dziedziczy po .NET-owym \href{https://msdn.microsoft.com/en-us/library/system.linq.expressions.expressionvisitor(v=vs.110).aspx}{\texttt{ExpressionVisitor}}. Służy ona do generowania kluczowych części zapytania oraz parametrów, które dane zapytanie będzie wykorzystywać.

Argumentami każdej z metod, które będą nadpisywane, są obiekty dziedziczące po \texttt{Expression}. Na ich podstawie budowany jest napis w klasie \href{https://msdn.microsoft.com/pl-pl/library/system.text.stringbuilder(v=vs.110).aspx}{\texttt{StringBuilder}}, który po zakończeniu odwiedzania wyrażenia zostaje przekazany do omawianej już implementacji \texttt{QueryModelVisitorBase}.

\subsection{Metoda \texttt{VisitQuerySourceReference}}
TODO.

\subsection{Metoda \texttt{VisitSubQuery}}
TODO.

\subsection{Metoda \texttt{VisitBinary}}
TODO.

\subsection{Metoda \texttt{VisitConditional}}
TODO.

\subsection{Metoda \texttt{VisitConstant}}
TODO.

\subsection{Metoda \texttt{VisitMember}}
TODO.

\subsection{Metoda \texttt{VisitMethodCall}}
TODO.

\subsection{Metoda \texttt{VisitNew}}
TODO.

\subsection{Metoda \texttt{VisitUnary}}
TODO.

\section{Czynności wykonywane po budowie zapytania}
Potrafiąc zbudować dowolne zapytanie SQL-owe na podstawie zapytania LINQ, pozostaje już tylko kwestia wykonania go i zwrócenia wyniku w postaci obiektowej. Autor pracy postanowił użyć do tego celu bibliotek \href{}{Npgsql} (połączenie z bazą PostgreSQL, wykonywanie zapytania) oraz \href{}{Dapper} (rozszerzenie umożliwiające wygodne mapowanie relacyjno-obiektowe).

Z użyciem tych dwóch bibliotek, wykonanie zapytania i przetłumaczenie wyniku do postaci obiektowej jest bardzo proste - biblioteka Dapper rozszerza obiekt \texttt{DbConnection}, służący do łączenia z bazą danych, o metody \texttt{Query} i \texttt{Query<T>}, które umożliwiają automatyczne mapowanie wyniku do postaci obiektowej. Poniżej fragment kodu z przykładowej implementacji \texttt{IQueryExecutor}:

\begin{lstlisting}
var commandData = SomeQueryModelVisitor.GenerateQuery(queryModel);

var s = commandData.Statement;
var p = commandData.Parameters;

return connection.Query<SomeClass>(s, p);
\end{lstlisting}

W tym miejscu zmienna \texttt{result} jest typu \texttt{IEnumerable<T>}, gdzie \texttt{T} jest typem obiektu, który zwraca oryginalne zapytanie LINQ (jest ono typu \texttt{IQueryable<T>)}. Dapper wymaga, aby nazwy kolumn w relacjach mapowanych na obiekty były takie same, jak nazwy właściwości w klasie modelowej - stąd, podczas budowy zapytania, każda kolumna została przemianowana na nazwę odpowiadającej właściwości w klasie z użyciem słowa kluczowego \texttt{AS}.

Dapper niestety nie radzi sobie z typami anonimowymi ze względu na to, że nie istnieje publiczny konstruktor obiektów tego rodzaju. Obejściem tego problemu jest metoda rozszerzająca \texttt{QueryAnonymous<T>}. OPIS TODO.